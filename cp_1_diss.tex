\documentclass[11pt, oneside]{article}   	% use "amsart" instead of "article" for AMSLaTeX format
\usepackage{geometry}
\usepackage[linesnumbered]{algorithm2e}
\geometry{letterpaper}
\usepackage{graphicx}
\usepackage{amsmath, amsthm, amssymb}
\usepackage{parskip}
\usepackage{titlesec}
\usepackage{enumerate}
\graphicspath{ {images/} }

\newtheorem*{theorem}{Theorem}
\newtheorem*{lem}{Lemma}
%SetFonts
\titleformat{\paragraph}
{\normalfont\normalsize\bfseries}{\theparagraph}{1em}{}
\titlespacing*{\paragraph}
{0pt}{3.25ex plus 1ex minus .2ex}{1.5ex plus .2ex}
\title{CS340 - Project Description and Analysis}
\author{Yutong Li, Tianming Xu, Jiaping Wang}
\date{\today}							% Activate to display a given date or no date

\begin{document}
\maketitle

\section{Description}
\subsection{Notation}
%Before we start talking about our actual algorithm, since we have a lot of types of data in this problem, we'd like to denote them to make the paper be more readable and concise.\\
Let $S$ be the set of students, with $|S|=n$. Each student will have a list of class requests, so let $l_{s_i}$ to represent the $i^{th}$ student's list of class requests. We denote the set of teachers by $P$ and its order by $m$. Let $C$ be the set of classes, and so we have $|C|=2m$ classes. Denote the set of classrooms by $R$, with $|R|=y$, and the set of time slots as $T$, with $|T|=x$. 
\subsection{Algorithm Description}
Given $S, P, C, R, T$, we determine a class schedule with following rule: schedule as many students as possible into each class in decreasing order of the class' popularity. \par 
Start off by sorting the classes in decreasing order of popularity, which is defined by the number of students who signed up for each class. Sort the classrooms in decreasing order of capacity. 

Then, we can begin pairing classes with classrooms and time periods. Starting with the most popular class $c_1$, we try scheduling this class in the biggest classroom $r_1$ during the first time slot $t_1$. Find the teacher $p_1$ who teaches $c_1$. If $p_1$ has no conflict during this time, put the class-room-time combination into the final schedule. Otherwise, add the assignment $(r_1, t_1)$ as a skipped slot and move on to the next time slot. Once this class is scheduled, mark both the teacher and classroom as unavailable during this time period. Repeat this process for the next most popular class in the next time slot, except that if there are previously skipped slots, we try scheduling the class to the first skipped slot first. Otherwise, try scheduling the class in the next largest classroom during its earliest time slot, checking for conflicts with teacher. When all time slots for a classroom has class assignments, we move on to the next largest classroom. Keep doing so until either all classrooms are filled at all times, or all $2m$ classes has been scheduled.\\
\section{Pseudocode}
\begin{algorithm}[H]
\SetKwFunction{Schedule}{Schedule}
\SetKwProg{Fn}{Function}{}{}
\Fn{\Schedule{S, P, C, R, T}}{
\For{all $c\in C$}{Count the number of students who signed up for $c$}
Sort $C$ in decreasing order of popularity.\\
Sort $R$ in decreasing order of capacity.\\
$k, j\gets 0$\\
$skippedSlots\gets$ empty\\
 \While{$j < y$ and $k<2m$}{
 	Choose classroom $R_j$\\
	$i\gets 0$\\
    \While{$i < x$  and $k<2m$}{
    	Choose time $T_i$\\
 	\uIf{skippedSlots is empty}{
        	\uIf{ teacher $P_{C_k}$ who teaches $C_k$ is available in $T_i$}{
        		$Schedule[R_j][T_i]\gets C_k$\\
		$i\gets i+1, k\gets k+1$\\
		append $T_i$ to $occupied[P_{C_k}]$}
        	\uElse{
        		append $(R_j, T_i)$ to $skippedSlots$\\
		$i\gets i+1$
    		  }
       	 }
	 \uElse{
	   pop the first element in $skippedSlots$\\
	   \uIf{ teacher $P_{C_k}$ has a conflict with the skipped slot}{
	    push the slot back to $skippedSlots$\\
	    repeat lines $18-24$
	   }\uElse{
	   	$Schedule[skippedSlot]\gets C_k$\\
		$k\gets k+1$}
         }
        	}
	$j\gets j+1$
}

\For{each scheduled class}{
	choose students randomly \\and check whether there's time conflict for the student before adding
}
\Return $Schedule$
}
\end{algorithm}

\section{Time Analysis}
\subsection{Data Structure}
The class requests of each student can be stored as a linked list. Constructing the lists in $S$ takes $O(n)$ and accessing a student's class choices takes $O(1)$. Furthermore, it takes $O(m)$ to traverse all class requests and initialize the array $K$. Sorting $K$ with the standard quicksort requires $O(m\log m)$.\par
The instructor information $P$ for classes can be stored in an array indexed by the classes. Thus initialization of $P$ takes $O(m)$, and retrieving the teacher of a given class $C_k$ takes $O(1)$. Similarly, classroom capacities can be stored in an array. Construction of $R$ takes $O(y)$ and sorting $R$ takes $O(y\log y)$.\par
Information about teachers' assignments can be stored in a dictionary indexed by the teachers. Each time a class is scheduled, append the time $t_i$ to the teacher's assignments. Because each teacher teaches at most two classes, the size of a teacher's assignments cannot exceed $1$. Thus, checking whether $P_{C_k}$ has conflict during time $T_i$ takes $O(1)$. Construction of the dictionary takes $O(m)$.\par
The skipped slots due to teachers' time conflicts can be stored in a FIFO queue. Therefore, pushing a slot into the queue, checking if it's empty, as well as popping its first element can be done in constant time. Since each teacher teaches at most two classes, each teacher can have at most $1$ conflict and so construction requires $O(m)$.\par
To store the final schedule, we can create a dictionary of dictionaries. The outer dictionary has $r_j\in R$ as keys, and for each $r_j$ we construct a dictionary of size $x$, where the $i$-th value in the array indicates whether $r_j$ is available during time $t_i$. Constructing the nested dictionary takes $O(xy)$. Accessing or changing each value can be done in $O(1)$.\par
Consider the time for choosing students for each class already been scheduled. Since in optimal situation all students and all class can be scheduled, thus in sum there're $4n$ students to be scheduled. The choosing students algorithm is based on built-in random selection, and thus takes $O(1)$ for each student. Further more, for each students, there's a need to check whether there're conflict in this student's schedule, which takes $O(1)$. In sum, choosing students have a $O(n)$ complexity.


\subsection{Time Complexity}
Using the data structures described above, we know that, in total, initialization requires \[n+n+n\log n+y+y\log y+m+m+xy+n = O(m\log m+y\log y +xy+n).\] By construction of the algorithm, the {\it while} loop terminates when all classrooms are $filled$ during all time periods or $2m$ classes are scheduled. Therefore, the nested {\it while} loops runs $\min(xy, 2m)$ times. Therefore, the overall time complexity of the algorithm is 
\begin{align*}
T(n) &= O(m\log m+y\log y +xy+n)+O(\min(xy, 2m))\\
& = O(n+y\log y+m\log m +xy)
\end{align*}
Since $n$ dominates, we can say the time complexity of the algorithm is $O(n)$.

\section{Proof of Correctness}
\underline{\textit{Proof of Termination.}} By time analysis we know the {\it while} loop runs at most $\min(xy, 2m)$ times. Each time a class is scheduled, $k, j$ are incremented by $1$. Furthermore, each time $j$ reaches $x$, $i$ is also incremented by $1$. Therefore, one of the loop preconditions will be violated as $i$ and $k$ keeps increasing. Hence, the algorithm terminates.

\underline{\textit{Proof of Validity.}}
We claim that each class-room-time combination added to the final schedule is valid. \\ \par
\begin{lem}No two classes are scheduled in the same classroom at the same time.\end{lem}
\begin{proof}Suppose for a contradiction that $C_k$ and $C_l$ are both scheduled into the same slot, $(R_j, T_i)$. Without loss of generality, suppose $l>k$ in the sorted list of classes. By execution of the algorithm, we know $C_k$ gets added into the schedule before $C_l$. Then, when $C_k$ is added, we increment $i$ by $1$ so that all classes after $C_k$ don't get assigned to the same time. When $i$ exceeds $x$, we set $i$ back to $0$ and increment $j$. Therefore, by the time $C_l$ gets scheduled to $(R_j', T_i')$ we have:
\[ \begin{cases} j'\geq j \\i'>i& \text{,if } $j=j'$ \end{cases}\]
Therefore, it is impossible to have $j=j'$ and $i=i'$ at the same time. Hence, there does not exist two classes that conflict with each other. \end{proof}
\begin{lem}No teacher is teaching two classes at the same time.\end{lem}
\begin{proof}Suppose there is a teacher $p$ who is scheduled to teach $C_k$ and $C_l$ at the same time. Without loss of generality, suppose $C_k$ is more popular than $C_l$. Thus, when $C_l$ gets scheduled, $C_k$ has already been scheduled in room $R_j$ at time $T_i$. Thus, $occupied[p]$ will contain $T_i$.\par
However, before the schedule of $C_l$ is added, we check for $occupied[p]$. Since $T_i$ is in $occupied[p]$, $C_l$ will never be scheduled during the same time $T_i$. Therefore, we conclude this is impossible. 
\end{proof}
Since the returned schedule satisfies the two lemmas above, it does not contain any conflicts and hence is a valid schedule.
\section{Discussion}
Possible solution might come from following categories:  dynamic programming, graph algorithm, recursion, naive algorithm, and greedy algorithm. 
\par First consider dynamic programming and recursion. Both algorithms involve ideas of dividing and conquering. However, in scheduling problem, since rooms can only be used by single class in each time slots, dividing students into different groups and arrange schedules respectively is trivial. Thus, dynamic programming and recursions cannot apply to this problem.\par
Then suppose graph algorithm can tackle this problem. According to the property of all graph algorithm, there must exist a way to transform given inputs into a graph data model, such as DAG. In any graph model, limits are interpreted as weights of edge, while objects constrained by limit are interpreted as vertices. Consider following: since the performance of algorithm depends on how many classes students can take, consequently, students are defined as vertices. Furthermore, the order of schedule has no effect on the performance, thus the graph is undirected. As stated above, limits are interpreted as edges, and according to the description of the problem, room size, teachers, time slots are all restrictions the algorithm needs to fulfill. Nevertheless, each edge can only pair up with a single restriction. Hence, an undirected graph cannot represent scheduling problem.\par
Naive algorithm aims to enumerate all possible schedules and compares their performance in order to get an optimal output. Yet, restrictions exist among given input, and thus requires large amount of collisions check. As a result, naive algorithm cannot given an acceptable time complexity.\par
Greedy algorithm guarantees optimal result for each iteration, in this case the number of class students can take. Accordingly, greedy algorithm always needs to ensure the most popular class is correctly arranged. In other words, the most popular class need to pair up with the largest room in order to get as most students as possible enrolled.
\par What is more, limits also exist on room size and time slots. Owing to the fact that a room can only be arranged to one time slot at a time, algorithm needs to schedule classes not only regarding room size but also guarantee that time conflict will not occur among popular classes.
\par Complications come mainly from restrictions on teachers and time slots. A simple corner case is that a teacher teaches two classes $c_i,c_j$, $c_i$ is more popular in $c_j$, but the algorithm automatically arrange them into the same time slot. A natural way to solve this conflict is to simply skip that time slot and move downward. However, this will result in an empty room at that time slot. To make full use of each room, a skipped room marker is needed. Whenever the algorithm skipped a room, the marker will keep track of that room, and arrange the next popular to that time slot.
\par To conclude, greedy algorithm will give an near-optimal result and can be done in an acceptable time complexity.
 
 \section{Experimental Analysis}
 \subsection{Time Analysis:}
As we stated above, our expectation on run time is $O(n)$. The result of our experiment shows that it indeeds have a linear run time. The algorithm is run on input generated by $\text{make\_random.pl}$ with different number of students. Specifically, input is generated with 50 rooms, 360 classes, 10 time 
slots fixed and the variable is number of students ranged from 50 to 5000. For each set of parameters, we generated 5 different input. Then for every set, we compute an average time, and create a graph where the x-axis represents the number of students and y-axis represents run time. After curve fitting, we find that it's linear. To validate this mathematically, we compute the $\frac{\text{number of students}}{\text{run time}}$ for each set of input. The result shows that most of the ratio lies around $800$ and the standard deviation is $38.57$. Observe that the main deviation occurs when the student size is small, $ie$ when there're 1000 students. We trust that this deviation is caused by the less frequent occurrence  of conflicts when there're fewer students. We also repeat the experiment with 20 time slots, and curve fitting generates a similar result.
\subsection{Quality Analysis:}
Within each set of input, our algorithm shows a stable performance. For instance, for 5000 students, 10 times slots, 50 rooms, 360 classes, the standard deviation on $student$  $preference$  $value$ is 0.0271174. Furthermore, the most significant stand deviation occurs when there're 50 students, which is 1.2649, still implying a stable performance. 
\\Then consider the performance on $student$ $preference$ $value$ for sets of input where number of rooms, classes time slots fixed and student size is the variable. We choose the lower bound of $student$ $preference$ $value$ for each set of input. First, consider there're 10 time slots, 50 rooms, 360 classes. For this set of input, our student size ranges from 1000 to 5000. Computing $\frac{\text{student preference value}}{\text{best Case Student Value}}$, we find that for different size of students, the ratio is almost the same, with a standard deviation of 0.0009. Thus, the average of these ratios can represent the performance of our algorithm on these set of input. With an average of $0.86308125$, our algorithm can arrange $86\%$ of students for their preferred classes. Then we repeat the experiment on different sets of input with 20 time slots, 50 rooms, 360 classes, and again student size ranges from 1000 to 5000. For these set of data, we apply the same analysis strategy and get an average of $0.93096875$ ratio. This ratio means that for 20 time slots, 50 rooms, 360 classes, we can arrange $93\%$ of students for their preferred classes.\\
\\We think the difference between the two different ratios result from different number of time slots. For the second set of data, there're 20 time slots, indicating more slots for classes and also a less occurrence of time conflicts. Under such a circumstance, our algorithm can arrange more classes and will thus generate a better performance.
\end{document}

